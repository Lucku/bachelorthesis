
\chapter{Fazit und Ausblick}

Diese Arbeit hat sich mit der Frage auseinandergesetzt, ob Intels Software Guard Extensions dafür geeignet sind, die datenverarbeitende Komponente in einem sicheren Datenbanksystem zu bilden, oder ob es sich doch nur um einen sicheren Container zum Speichern von Schlüsseln handelt. Um jener Problematik auf den Grund zu gehen, wurden Untersuchungen der technischen Möglichkeiten, sowie Leistung der Technologie unternommen.

Im Vergleich zu anderen Ansätzen sicherer Datenbanksysteme, hat der Einsatz von Intel \ac{SGX} einen großen Vorteil bezüglich der Infrastruktur. Da es auf jedem neueren Intel Prozessor zur Verfügung steht, ist es nicht nötig, zusätzliche Komponenten in das System zu integrieren. Außerdem bietet Intel Werkzeuge zur einfachen Anwendungsentwicklung an, welche es einfach machen, beliebige datenverarbeitende Operationen in ein sicheres Umfeld zu verlagern. Die Unterstützung von Befehlssatzerweiterungen zur Parallelisierung sorgt dafür, dass auch hochleistungsfähige Algorithmen umgesetzt werden können. Die Untersuchungsergebnisse haben gezeigt, dass die Geschwindigkeit der Datenverarbeitung kaum unter der Nutzung einer sicheren Enclave leidet. Insofern während der Entwicklung ein achtsamer Programmstil an den Tag gelegt und stets auf größeren Datenmengen gearbeitet wird, ist der Overhead an Rechenaufwand im Vergleich zu einem normalen Programmablauf vernachlässigbar.

Den genannten Vorteilen steht als limitierender Faktor der begrenzt zur Verfügung stehende Speicherplatz gegenüber. Der gesicherte, vom Prozessor reservierte Speicher ist aktuell unabhängig vom eingebauten Hauptspeicher auf 128 MB beschränkt, wovon etwa 90 MB zwischen den Enclaves des Nutzers aufgeteilt werden dürfen. Dazu kommt, dass die genaue Größe einer Enclave zum Zeitpunkt ihrer Initialisierung bekannt, und entsprechend vom Entwickler konfiguriert werden muss. Je nach benötigter Kapazitäten im produktiven Einsatz kann sich eine ständige Ein- und Auslagerung von Daten als unvorteilhaft bezüglich der Verarbeitungsdauer herausstellen. Aus Sicht der Datensicherheit birgt \ac{SGX} ein weiteres Problem. Auch wenn sämtliche Abhängigkeiten vom Betriebssystemkern oder Hypervisor aufgelöst sind, sorgt das Verschieben von einer hohen Bandbreite an datenverarbeitender Funktionalität in eine Enclave zugleich dafür, dass die Trusted Computing Base sehr umfangreich ausfällt. Dies begünstigt das Auftreten möglicher Sicherheitslücken. Intel überlässt derweil dem Programmierer, einen fehlerfreien Code zu produzieren, welcher etwa Seitenkanalangriffe ausschließt.

Es lässt sich zusammenfassen, dass das es Sinn macht, Intel \ac{SGX} als Teil eines sicheren Datenbanksystems einzusetzen. Je nach dessen Ausmaß steht jedoch der verfügbare Speicherplatz als Limitation gegenüber. Kompressionsalgorithmen wie VByte und die Lauflängenkodierung können genutzt werden, um die Dichte der Daten zu erhöhen, und einen wichtigen Vorteil zu schaffen. Jene Algorithmen können ohne schwerwiegende Leistungseinschränkungen in einer Enclave durchgeführt werden. Dem Problem kann weiterhin durch die gezielte Ein- und Auslagerung von Zwischenergebnissen in den nicht vertraulichen Speicher in Kombination mit eigenschaftserhaltenden Verschlüsselungsschemata entgegengewirkt werden. Andererseits können wiederum die Möglichkeit der dynamischen Allokation von Enclavespeicher und die weiteren Änderungen, welche im Zuge des Nachfolgers \ac{SGX} 2 in Kraft treten werden, neue Lösungsansätze liefern. Daraus, inwiefern die Effektivität aller jener Maßnahmen zu bewerten ist, ergeben sich Fragestellungen für weiterführende Untersuchungen. Als noch sehr junge und aufkommende Technologie bleibt es abzuwarten, in welche Richtung Intel mit seiner Technologie in Zukunft gehen wird und welche Ansätze entworfen werden, um von \ac{SGX} in sicherheitskritischen Systemen Gebrauch zu machen.