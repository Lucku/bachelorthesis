
\chapter{Fazit und Ausblick}

Diese Arbeit hat sich mit der Frage auseinandergesetzt, ob Intels Secure Guard Extensions dafür geeignet sind, die datenverarbeitende Komponente in einem sicheren Datenbanksystem zu bilden, oder ob es sich doch nur um einen sicheren Container zum Speichern von Schlüsseln handelt. Um jener Problematik auf den Grund zu gehen, wurden Untersuchungen der technischen Möglichkeiten, sowie  Leistung der Technologie unternommen.

Im Vergleich mit anderen Ansätzen sicherer Datenbanksysteme, hat der Einsatz von Intel SGX einen großen Vorteil bezüglich der notwendigen Infrastruktur. Da es auf jedem neueren Intel Prozessor zur Verfügung steht, ist es nicht nötig, zusätzliche Komponenten in das System zu integrieren. Außerdem bietet Intel Werkzeuge zur einfachen Anwendungsentwicklung an, welche es einfach machen, beliebige datenverarbeitende Operationen in ein sicheres Umfeld zu verlagern. Die Unterstützung von Intels Befehlssatzerweiterungen zur Parallelisierung sorgt dafür, dass auch hochleistungsfähige Algorithmen umgesetzt werden können. Als besonders positiver Aspekt hat sich gezeigt, dass die Geschwindigkeit der Datenverarbeitung kaum unter der Nutzung einer sicheren Enclave leidet. Insofern während der Entwicklung ein achtsamer Programmstil an den Tag gelegt und stets auf größeren Datenmengen gearbeitet wird, ist der Overhead an Rechenaufwand im Vergleich zu einem normalen Programmablauf nicht signifikant. 

Den genannten Vorteilen steht als limitierender Faktor der begrenzt zur Verfügung stehende Speicherplatz im sicheren Bereich gegenüber. Der sichere, vom Prozessor reservierte Speicher ist aktuell unabhängig vom eingebauten Hauptspeicher auf 128 MB beschränkt, wovon etwa 90 MB zwischen den Enclaves des Nutzers aufgeteilt werden dürfen. Dazu kommt, dass die genaue Größe einer Enclave zum Zeitpunkt ihrer Initialisierung bekannt, und entsprechend vom Entwickler konfiguriert werden muss. Da im produktiven Einsatz oftmals wesentlich mehr Kapazitäten vonnöten sein können, kann sich eine ständige Ein- und Auslagerung von Daten als unvorteilhaft bezüglich der Verarbeitungsdauer herausstellen. Aus Sicht der Datensicherheit birgt SGX ein weiteres Problem. Auch wenn SGX die Abhängigkeiten zum Betriebssystemkern oder Hypervisor auflöst, sorgt das Verschieben von einer hohen Bandbreite an datenverarbeitender Funktionalität in die Enclave zugleich dafür, dass die Trusted Computing Base sehr umfangreich ausfällt. Dies begünstigt das Auftreten möglicher Sicherheitslücken. Intel überlässt derweil dem Programmierer, einen fehlerfreien Code zu produzieren, welcher etwa Seitenkanalangriffe ausschließt.

Insgesamt haben die Ergebnisse gezeigt, dass Intel SGX das Potenzial besitzt, als Teil eines sicheren Datenbanksystems eingesetzt zu werden. Der begrenze Speicherplatz für Enclaves stellt jedoch das größte Hindernis für eine Nutzung in einem größeren Maßstab dar. Die Möglichkeit der dynamischen Allokation von Speicher der Enclaves und die weiteren Änderungen, welche im Zuge von SGX 2 in Kraft treten, können bereits neue Lösungsansätze liefern. Andererseits kann dem Problem durch den gezielten Einsatz von Ein- und Auslagerung von Zwischenergebnissen in den unsicheren Speicher sowie einer Integration von eigenschaftserhaltenden Verschlüsselungsschemata entgegengewirkt werden. Es ergeben sich somit weitere wichtige Fragen für weiterführende Untersuchungen. Als noch sehr junge und aufkommende Technologie bleibt es abzuwarten, welche Ansätze entwickelt werden, um in sicherheitskritischen System von Intel SGX Gebrauch zu machen.