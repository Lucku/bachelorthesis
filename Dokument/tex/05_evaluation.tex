
\chapter{Evaluierung}
%TODO intro

\begin{itemize}
	\item Einheit: MIOPS
\end{itemize}

\section{Konzept}
%TODO Was ist in unserem Kontext sinnvoll zu untersuchen und warum?

\begin{itemize}
	\item jeweils Bulkverarbeitung anstatt einzeln
	\item jegliche Optimierungen wurden zu Zwecken der Gleichheit im Compiler deaktiviert
\end{itemize}

\subsection{Untersuchungen}
\subsection{Testumgebung}
\begin{itemize}
	\item Intel Performance Library Cryptography eingesetzt
	\item Anwendungen mit Intel Compiler erstellt
\end{itemize}

\section{Enclave interne Untersuchungen}
\subsubsection{Overhead beim Kopieren der Daten in und aus Enclave}

\begin{itemize}
	\item \textbf{(1)} reines Kopieren vs. kein Kopieren
	\begin{figure}[h]
		\includegraphics[width=\linewidth]{img/Eval1.pdf}
		\centering
		\caption{Vergleich Kopieren der Daten in die Enclave und kein Kopieren mittels user\textunderscore check}
		\label{fig:eval1}
	\end{figure}
	
	\begin{itemize}
		\item außerhalb der Enclave: konstante Steigung der Datenrate um 0,45MIOPS/Byte = 0,02ms/Byte
		\item innerhalb der Enclave nimmt die Steigung der Funktion ständig ab, der Aufwand zum Kopieren der Daten wird ständig höher als die Ersparnis des häufigen Springens in die Enclave
	\end{itemize}
	
	\item \textbf{(2)} Kopieren/Entschlüsselung/Verschlüsselung/Kopieren vs. ohne Kopieren \textbf{(13)/(14)} 
	\begin{figure}[h]
		\includegraphics[width=\linewidth]{img/Eval2.pdf}
		\centering
		\caption{Vergleich Ent- und Verschlüsselung mit und ohne Kopiervorgang}
		\label{fig:eval2}
	\end{figure}
	
\end{itemize}

\section{Datenverarbeitung in der Enclave}
\subsubsection{Allgemein}

\begin{itemize}
	\item \textbf{(3)} Iteration über Datenmengen 
	\begin{figure}[h]
		\includegraphics[width=\linewidth]{img/Eval3.pdf}
		\centering
		\caption{Vergleich Iteration über Datenmengen außerhalb und innerhalb der Enclave}
		\label{fig:eval3}
	\end{figure}
\end{itemize}
\subsubsection{Einfache Kompressionsverfahren}
\begin{itemize}
	\item \textbf{(4)} VByte Kompression außerhalb vs. in Enclave
	\begin{figure}[h]
		\includegraphics[width=\linewidth]{img/Eval4.pdf}
		\centering
		\caption{Vergleich VByte Kompression außerhalb und innerhalb der Enclave}
		\label{fig:eval4}
	\end{figure}
	
	\item \textbf{(5)} VByte Dekompression außerhalb vs. in Enclave
	\begin{figure}[h]
		\includegraphics[width=\linewidth]{img/Eval5.pdf}
		\centering
		\caption{Vergleich VByte Dekompression außerhalb und innerhalb der Enclave}
		\label{fig:eval5}
	\end{figure}
	
	\item \textbf{(6)} VByte Dekompression und Kompression außerhalb vs. in Enclave
	\begin{figure}[h]
		\includegraphics[width=\linewidth]{img/Eval6.pdf}
		\centering
		\caption{Vergleich VByte Dekompression und Kompression außerhalb und innerhalb der Enclave}
		\label{fig:eval6}
	\end{figure}
	
	\item \textbf{(7)} Lauflängenkodierung außerhalb vs. in Enclave
	\begin{figure}[h]
		\includegraphics[width=\linewidth]{img/Eval7.pdf}
		\centering
		\caption{Vergleich Lauflängenkodierung außerhalb und innerhalb der Enclave}
		\label{fig:eval7}
	\end{figure}

	\item \textbf{(8)} Lauflängendekodierung außerhalb vs. in Enclave
	\begin{figure}[h]
		\includegraphics[width=\linewidth]{img/Eval8.pdf}
		\centering
		\caption{Vergleich Lauflängendekodierung außerhalb und innerhalb der Enclave}
		\label{fig:eval8}
	\end{figure}

	\begin{itemize}
		\item großer Versatz dadurch, dass sehr große Buffer für die Dekodierung reserviert werden, haben 6-fache Größe, Herauskopieren nimmt viel Zeit in Anspruch
	\end{itemize}

	\item \textbf{(9)} Lauflängensummenbildung außerhalb vs. in Enclave
	\begin{figure}[h]
		\includegraphics[width=\linewidth]{img/Eval9.pdf}
		\centering
		\caption{Vergleich Lauflängenkodierung und Summenbildung außerhalb und innerhalb der Enclave}
		\label{fig:eval9}
	\end{figure}

\end{itemize}
\section{Kryptographische Verarbeitung}
\begin{itemize}
	
	\item \textbf{(10)} Entschlüsselung und Dekomprimierung außerhalb vs. in Enclave
	\begin{figure}[h]
		\includegraphics[width=\linewidth]{img/Eval10.pdf}
		\centering
		\caption{Vergleich Entschlüsselung und Dekomprimierung außerhalb und innerhalb der Enclave}
		\label{fig:eval10}
	\end{figure}
	
	\item \textbf{(11)} Komprimierung und Verschlüsselung außerhalb vs. in Enclave
	\begin{figure}[h]
		\includegraphics[width=\linewidth]{img/Eval11.pdf}
		\centering
		\caption{Vergleich Komprimierung und Verschlüsselung außerhalb und innerhalb der Enclave}
		\label{fig:eval11}
	\end{figure}
	
	\item \textbf{(12)} Entschlüsselung und Verschlüsselung außerhalb vs. in Enclave
	\begin{figure}[h]
		\includegraphics[width=\linewidth]{img/Eval12.pdf}
		\centering
		\caption{Vergleich Entschlüsselung und Verschlüsselung außerhalb und innerhalb der Enclave (mit und ohne Kopieren)}
		\label{fig:eval12}
	\end{figure}
\end{itemize}
\section{Kombinierte Verarbeitungsschritte}

\begin{itemize}
	\item \textbf{(13)} Entschlüsseln/Dekomprimieren/Lauflängenkodierung/Komprimierung/Verschlüsseln
	\begin{figure}[h]
		\includegraphics[width=\linewidth]{img/Eval13.pdf}
		\centering
		\caption{Vergleich Verarbeitungsabfolge außerhalb und innerhalb der Enclave}
		\label{fig:eval13}
	\end{figure}

\end{itemize}

\section{Fazit}
%TODO notes