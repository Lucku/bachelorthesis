
\begin{abstract}
\begin{otherlanguage}{english}
\section*{Abstract}
Intel Software Guard Extensions (SGX) are the latest technology of the processing hardware vendor to allow the work on data in a protected region in memory. Therefore it is organized in Enclaves, independent and secure containers that can be individually defined and used by common applications. This work focuses on the question, if it is suggestive to put the whole query processing of an In-memory database inside such an container, leveraging the security guaranties and easy integration into an existing infrastructure. Furthermore, the usage of lightweight data compression techniques as a tool to increase the processing performance is considered as well. General investigations included the analysis of constraints, which SGX involves from a technical and development perspective. To evaluate the performance of data processing inside an Enclave, several procedures including compression and encryption operations were tested within performance benchmarks. On an implementation level, the corresponding functions were called on an increasing size of input data, both in a secure and conventional application environment. A comparison of the results in each procedure showed that the processing among SGX can nearly reach the same performance as if executed in a usual manner. Together with the freedom in development, this leads to the conclusion that there are no limitations regarding the execution of processing operations themselves. On the other hand, the restriction of available memory inside the Enclave remains as a problem preventing the use of SGX as part of database systems, which have to deal with workloads of greater dimension.
\end{otherlanguage}
\end{abstract}